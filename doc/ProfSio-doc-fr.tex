% !TeX TXS-program:compile = txs:///arara
% arara: lualatex: {shell: yes, synctex: no, interaction: batchmode}
% arara: lualatex: {shell: yes, synctex: no, interaction: batchmode}
% arara: lualatex: {shell: yes, synctex: no, interaction: batchmode}

\documentclass[french,a4paper,11pt]{article}
\usepackage[margin=2cm,includefoot]{geometry}
\def\TPversion{0.1.2}
\def\TPdate{3 juillet 2023}
%\usepackage[utf8]{inputenc}
%\usepackage[T1]{fontenc}
\usepackage{amsmath,amssymb}
\usepackage{ProfSio}
\usepackage{awesomebox}
\usepackage{fontawesome5}
\usepackage{footnote}
\makesavenoteenv{tabular}
\usepackage{enumitem}
\usepackage{wrapstuff}
\usepackage{lipsum}
\UseTblrLibrary{booktabs}
\usepackage{fancyvrb}
\usepackage{fancyhdr}
\fancyhf{}
\renewcommand{\headrulewidth}{0pt}
\lfoot{\sffamily\small [ProfSio]}
\cfoot{\sffamily\small - \thepage{} -}
\rfoot{\hyperlink{matoc}{\small\faArrowAltCircleUp[regular]}}

%\usepackage{hvlogos}
\usepackage{hologo}
\providecommand\tikzlogo{Ti\textit{k}Z}
\providecommand\TeXLive{\TeX{}Live\xspace}
\providecommand\PSTricks{\textsf{PSTricks}\xspace}
\let\pstricks\PSTricks
\let\TikZ\tikzlogo
\newcommand\TableauDocumentation{%
	\begin{tblr}{width=\linewidth,colspec={X[c]X[c]X[c]X[c]X[c]X[c]},cells={font=\large\sffamily}}
		{\LaTeX} & {\hologo{pdfLaTeX}} & {\hologo{LuaLaTeX}} & {\TikZ} & {\TeXLive} & {\hologo{MiKTeX}} \\
	\end{tblr}
}

\usepackage{hyperref}
\urlstyle{same}
\hypersetup{pdfborder=0 0 0}
\setlength{\parindent}{0pt}
\definecolor{LightGray}{gray}{0.9}

\usepackage{babel}
\AddThinSpaceBeforeFootnotes
\FrenchFootnotes

%\usepackage{listings}

\usepackage{newverbs}
\newverbcommand{\motcletex}{\color{cyan!75!black}}{}
\newverbcommand{\packagetex}{\color{violet!75!black}}{}

\usepackage[most]{tcolorbox}
\tcbuselibrary{listingsutf8}
\newtcblisting{DemoCode}[1][]{%
	enhanced,width=0.95\linewidth,center,%
	bicolor,size=title,%
	colback=cyan!2!white,%
	colbacklower=cyan!1!white,%
	colframe=cyan!75!black,%
	listing options={%
		breaklines=true,%
		breakatwhitespace=true,%
		style=tcblatex,basicstyle=\small\ttfamily,%
		tabsize=4,%
		commentstyle={\itshape\color{gray}},
		keywordstyle={\color{blue}},%
		classoffset=0,%
		keywords={},%
		alsoletter={-},%
		keywordstyle={\color{blue}},%
		classoffset=1,%
		alsoletter={-},%
		morekeywords={center,justify,\lipsum},%
		keywordstyle={\color{violet}},%
		classoffset=2,%
		alsoletter={-},%
		morekeywords={\MPMPlaceTache,\MPMPlaceNotice,\MPMPlaceDuree,GrapheMPM,TableKarnaugh,\KarnaughCasesResult,\KarnaughBlocRegroup,\MPMPlaceTaches,\MPMPlaceDurees,GrapheTikz,\GrphPlaceSommets,\GrphTraceAretes,\tikzset,\DiagrammeSagittal,\draw,\DiagrammeSagittalCompo,\TableVerite},%
		keywordstyle={\color{green!50!black}},%
		classoffset=3,%
		morekeywords={CouleurDurees,CouleurFleches,LargeurCases,Epaisseur,Police,CouleurDates,CouleurBords,NoirBlanc,Grille,DecalHorizDeb,DecalVertDeb,DecalHorizFin,DecalVertFin,Coude,SensCoude,Unite,Variables,Swap,Aide,CouleurCases,Decalage,Couleur,Type,Legende,PosVarLaterale,CouleurLegende,CouleurSommets,TypeSommets,Unite,CouleurFT,DimensionSommets,PositionFleches,EchelleFleches,TypeFleche,Droit,Milieu,AngleGauche,AngleDroite,Boucle,GrphStyleArc,GrphStyleSommet,Poids,GrphStylepoids,DistElem,DistEns,LargEns,NomAppli,CouleurE,CouleurAppli,CouleurF,CouleursFleches,TypeFleche,Epaisseur,Labels,Ensembles,PosLabels,PoliceLabels,Offset,NomApplis,CouleursAppli,VF,LargeursColonnes,CouleurEnonce,CodeAvant,CodeApres},%
		keywordstyle={\color{orange}}
	},%
	#1
}

\tcbset{vignettes/.style={%
	nobeforeafter,box align=base,boxsep=0pt,enhanced,sharp corners=all,rounded corners=southeast,%
	boxrule=0.75pt,left=7pt,right=1pt,top=0pt,bottom=0.25pt,%
	}
}

\tcbset{vignetteMaJ/.style={%
	fontupper={\vphantom{pf}\footnotesize\ttfamily},
	vignettes,colframe=purple!50!black,coltitle=white,colback=purple!10,%
	overlay={\begin{tcbclipinterior}%
			\fill[fill=purple!75]($(interior.south west)$) rectangle node[rotate=90]{\tiny \sffamily{\textcolor{black}{\scalebox{0.66}[0.66]{\textbf{MàJ}}}}} ($(interior.north west)+(5pt,0pt)$);%
	\end{tcbclipinterior}}
	}
}

\newcommand\Cle[1]{{\small\sffamily\textlangle \textcolor{orange}{#1}\textrangle}}
\newcommand\cmaj[1]{\tcbox[vignetteMaJ]{#1}\xspace}

\begin{document}

\setlength{\aweboxleftmargin}{0.07\linewidth}
\setlength{\aweboxcontentwidth}{0.93\linewidth}
\setlength{\aweboxvskip}{8pt}

\pagestyle{fancy}

\thispagestyle{empty}

\vspace{2cm}

\begin{center}
	\begin{minipage}{0.75\linewidth}
	\begin{tcolorbox}[colframe=yellow,colback=yellow!15]
		\begin{center}
			\begin{tabular}{c}
				{\Huge \texttt{ProfSio} [fr]}\\
				\\
				{\LARGE Des outils pour les Maths en BTS SIO.} \\
			\end{tabular}
			
			\bigskip
			
			{\small \texttt{Version \TPversion{} -- \TPdate}}
		\end{center}
	\end{tcolorbox}
\end{minipage}
\end{center}

\begin{center}
	\begin{tabular}{c}
	\texttt{Cédric Pierquet} ({\ttfamily c pierquet -- at -- outlook . fr})\\
	\texttt{\url{https://github.com/cpierquet/profsio}}
\end{tabular}
\end{center}

\vspace{0.15cm}

{$\blacktriangleright$~~Commandes spécifiques pour le programme de Mathématiques en BTS SIO\footnotemark\footnotetext{Brevet de Technicien Supérieur - Services Informatiques aux Organisations : \href{https://www.letudiant.fr/etudes/bts/bts-sio-services-informatiques-aux-organisations.html}{[Lien]} sur le site de L'Étudiant}.}

\vspace{0.15cm}

{$\blacktriangleright$~~Créer des diagrammes MPM\footnotemark\footnotetext{Méthode des Potentiels Métra : \href{https://fr.wikipedia.org/wiki/Méthode_des_potentiels_métra}{[Lien]} sur le site de Wikipedia} (Méthode des Potentiels Métra).}

\vspace{0.15cm}

{$\blacktriangleright$~~Créer des tables de Karnaugh avec mise en valeur (manuelle) des regroupements.}

\vspace{0.15cm}

{$\blacktriangleright$~~Créer des graphes simples ou des diagrammes sagittaux.

\vspace{0.15cm}

{$\blacktriangleright$~~Créer des tables de vérité (via \hologo{LuaLaTeX}) grâce au code du package \packagetex!luatruthtable!\footnotemark\footnotetext{Package \LaTeX{} : \href{https://ctan.org/pkg/luatruthtable}{[Lien]} sur le site du CTAN}.

\vspace{1cm}

\hfill
\begin{GrapheMPM}[LargeurCases=0.5cm]<scale=0.9>
	%NOTICE
	\MPMPlaceNotice(1,6.5)
	%SOMMETS
	\MPMPlaceTache(1,4)(Début)(0,0)
	\MPMPlaceTache(3.25,4)(COM)(0,0)
	\MPMPlaceTache(5.5,4)(LOG)(1,2)
	\MPMPlaceTache(5.5,2)(ECR)(1,1)
	\MPMPlaceTache(5.5,7)(MAT)(1,2{,}5)
	\MPMPlaceTache(7.75,7)(CABL)(2,4)
	\MPMPlaceTache(7.75,5.5)(ASS)(2,3{,}5)
	\MPMPlaceTache(10,4)(INST)(4,5)
	\MPMPlaceTache(12.25,4)(POST)(7,7)
	\MPMPlaceTache(14.5,4)(CONF)(8,8)
	\MPMPlaceTache(16.75,4)(Fin)(9,9)
%	%ARCS
	\MPMPlaceDuree{Début>COM,0}
	\MPMPlaceDuree{COM>MAT,1}\MPMPlaceDuree{COM>LOG,1}\MPMPlaceDuree{COM>ECR,1}
	\MPMPlaceDuree{MAT>CABL,1}\MPMPlaceDuree{MAT>ASS,1}
	\MPMPlaceDuree{LOG>INST,3}
	\MPMPlaceDuree[Coude]{ECR>POST,6}<near start>
	\MPMPlaceDuree[Coude]{CABL>CONF,4}<near start>
	\MPMPlaceDuree{ASS>INST,1{,}5}
	\MPMPlaceDuree{INST>POST,2}
	\MPMPlaceDuree{POST>CONF,1}
	\MPMPlaceDuree{CONF>Fin,1}
\end{GrapheMPM}
\hfill~

\hfill
\begin{TableKarnaugh}<scale=0.9,baseline=(current bounding box.center)>
	\KarnaughCasesResult{0,1,1,0,1,1,1,1}
	\KarnaughBlocRegroup[Type=Centre,Couleur=blue!75,Decalage=-1.5pt]{10}{32}
	\KarnaughBlocRegroup[Type=Gauche,Couleur=red!75,Decalage=-1.5pt]{00}{11}
	\KarnaughBlocRegroup[Type=Droite,Couleur=red!75,Decalage=-1.5pt]{40}{31}
\end{TableKarnaugh}
\hspace{1cm}
\begin{TableKarnaugh}[Variables=u/v/w,Swap,CouleurCases=lime]<scale=0.9,baseline=(current bounding box.center)>
	\KarnaughCasesResult*{1,1,1,1,1,0,0,0}
	\KarnaughBlocRegroup[Type=Centre,Couleur=blue!75,Decalage=-1.5pt]{00}{12}
	\KarnaughBlocRegroup[Type=Centre,Couleur=red!75,Decalage=-1.15pt]{01}{42}
\end{TableKarnaugh}
\hspace{1cm}
\TableVerite<baseline=c>{P,Q}{$P$,$Q$}{lognot*P,P*logand*Q,P*imp*Q}{$\lnot P$,$P \lor Q$,$P \Rightarrow Q$}
\hfill~

\vspace{0.5cm}

\hfill
\begin{GrapheTikz}[Unite=0.75cm,CouleurSommets={gray/blue},Epaisseur={very thick/thick},CouleurFleches=orange]<scale=0.9>
	\GrphPlaceSommets{(5,4)/A (2,2)/B (9,3)/C}
	\GrphTraceAretes{A/B}
	\GrphTraceAretes[AngleGauche]{C/A}
	\GrphTraceAretes[AngleDroite]{B/C}
	\GrphTraceAretes[Boucle=4]{A/45 B/135 C/-45}
\end{GrapheTikz}
\hfill~
\DiagrammeSagittal[Labels=false,E={a,b,c},F={A,C,H,P},Labels=false]{a/A,a/P,b/H,b/P,c/C}
\hfill~

%%\hfill{}\textit{Merci à Patrick Bideault pour ses retours et conseils !}

%\vfill
%
%\hrule
%
%\medskip
%
%\TableauDocumentation
%
%\medskip
%
%\hrule

\newpage

\phantomsection
\hypertarget{matoc}{}

\tableofcontents

\vfill

\newpage

\section{Historique}

\verb|v0.1.2|~:~~~~Clé \Cle{Offset} pour les diagrammes sagittaux + Diagrammes sagittaux de composées.

\verb|      |~:~~~~Ajout des tables de vérité (via \hologo{LuaLaTeX}).

\verb|v0.1.1|~:~~~~Mise à jour de la documentation + Diagrammes sagittaux.

\verb|v0.1.0|~:~~~~Version initiale.

\newpage

\section{Le package ProfSio}

\subsection{Introduction}

\begin{noteblock}
Le package \packagetex!ProfSio! propose quelques commandes pour travailler sur des points particuliers de Mathématiques enseignées en BTS SIO :

\begin{itemize}
	\item graphe d'ordonnancement par la méthode MPM ;
	\item tableau de Karnaugh à 3 variables ;
	\item graphes \textit{simples} orientés ou pondérés, des diagrammes sagittaux ;
	\item des tables de vérité (via \hologo{LuaLaTeX}).
\end{itemize}
\vspace*{-\baselineskip}\leavevmode
\end{noteblock}

\begin{warningblock}
Le code ne propose par de \og résolution \fg{} du graphe MPM, de \og simplification \fg{} d'expressions booléennes ou de représentation \og automatique \fg{} d'un graphe, il ne consiste \textit{qu'en} une mise en forme du graphe MPM, du tableau de Karnaugh ou du graphe.

\smallskip

Par contre, pour les tables de vérité, le code se charge de créer le tableau \underline{entièrement}, grâce aux données du package \packagetex!luatruthtable! (légèrement \textit{patchées} pour obtenir une alternance un peu plus homogène).
\end{warningblock}

\subsection{Chargement du package, packages utilisés}

\begin{importantblock}
Le package se charge, de manière classique, dans le préambule.

Il n'existe pas d'option pour le package, et \packagetex!xcolor! n'est pas chargé.
\end{importantblock}

\begin{DemoCode}[listing only]
\documentclass{article}
\usepackage{ProfSio}

\end{DemoCode}

\begin{noteblock}
\packagetex!ProfSio! charge les packages suivantes :

\begin{itemize}
	\item \packagetex!tikz!, \packagetex!pgffor!, \packagetex!xintexpr!, \packagetex!tabularray!, \packagetex!simplekv!, \packagetex!xstring! et \packagetex!listofitems! ;
	\item \packagetex!luacode! et \packagetex!nicematrix! (uniquement si le compilateur détecté est \hologo{LuaLaTeX}) ;
	\item les librairies \packagetex!tikz! :
	\begin{itemize}
		\item \motcletex!tikz.positioning!, \motcletex!tikz.babel!, \motcletex!tikz.calc! ;
		\item \motcletex!tikz.decorations.pathreplacing! et \motcletex!tikz.decorations.markings! ;
		\item \motcletex!tikz.shapes!, \motcletex!tikz.shapes.geometric!, \motcletex!tikz.arrows! et \motcletex!tikz.arrows.meta!.
	\end{itemize}
\end{itemize}

Il est compatible avec les compilations usuelles en \textsf{latex}, \textsf{pdflatex}, \textsf{lualatex} (obligatoire pour les tables de vérité !) ou \textsf{xelatex}.
\end{noteblock}

\subsection{Fonctionnement global}

\begin{tipblock}
Les environnements sont créés avec \TikZ, et la majorité des paramètres des tracés sont personnalisables :

\hfill{}couleurs ; dimensions ; polices.\hfill~
\end{tipblock}

\begin{noteblock}
Le choix a été fait de :

\begin{itemize}
	\item présenter l'ordonnancement par la méthode MPM, avec présentation des tâches \textit{fixée} ;
	\item limiter les tableaux de Karnaugh pour 3 variables, avec présentation \textit{fixée} ;
	\item de ne pas forcément proposer de modification de la présentation \textit{globale}.
\end{itemize}
\vspace*{-\baselineskip}\leavevmode
\end{noteblock}

\pagebreak

\section{Graphe d'ordonnancement par méthode MPM}

\subsection{Commande et fonctionnement global}

\begin{cautionblock}
L'environnement dédié à la création du graphe d'ordonnancement est \motcletex!GrapheMPM!.

C'est en fait un environnement \motcletex!tikzpicture! personnalisé.

\smallskip

Les commandes à utiliser dans l'environnement sont :

\begin{itemize}
	\item \motcletex!\MPMPlaceNotice! ;
	\item \motcletex!\MPMPlaceTache! ou \motcletex!\MPMPlaceTaches! ;
	\item \motcletex!\MPMPlaceDuree! ou \motcletex!\MPMPlaceDurees!.
\end{itemize}
\vspace*{-\baselineskip}\leavevmode
\end{cautionblock}

\begin{DemoCode}[listing only]
\begin{GrapheMPM}[clés]<options tikz>
	\MPMPlaceNotice(*)(coordonnées)
	\MPMPlaceTache(coordonnées)(Tâche)(Dates)
	\MPMPlaceTaches{ (coordA)(TâcheA)(DatesA) / (coordB)(TâcheB)(DatesB) / ... }
	\MPMPlaceDuree[clés]{TâcheA>TâcheB,durée}<options tikz>
	\MPMPlaceDurees[clés]{TâcheA>TâcheB,durée / TâcheC>TâcheD,durée }<options tikz>
\end{GrapheMPM}
\end{DemoCode}

\begin{DemoCode}[]
\begin{GrapheMPM}
	\MPMPlaceNotice(-2,2.15)
	\MPMPlaceTaches{ (0,0)(F)(2,4) / (3,1)(G)(5,7) / (6,0.5)(L)(9,9) }
	\MPMPlaceDurees{F>G,1 / G>L,2}
	\MPMPlaceDuree[Coude,SensCoude=VHV]{F.south>L.south,4}<near start>
\end{GrapheMPM}
\end{DemoCode}

\begin{tipblock}
Les tâches sont créées sous forme de \textit{tableau} et sont associées à des nœuds, nœuds qui servent ensuite à positionner les durées des tâches.
\end{tipblock}

\pagebreak

\subsection{Arguments et clés pour l'environnement}

\begin{DemoCode}[listing only]
\begin{GrapheMPM}[clés]<options tikz>
	%commandes
\end{GrapheMPM}
\end{DemoCode}

\begin{tipblock}
En ce qui concerne la création de l'environnement, les \Cle{clés} sont :

\begin{itemize}
	\item \Cle{CouleurDurees} := couleur des durée ; \hfill~défaut : \Cle{purple}
	\item \Cle{CouleurFleches} := couleur des arcs ; \hfill~défaut : \Cle{blue}
	\item \Cle{LargeurCases} := largeur des cases ; \hfill~défaut : \Cle{0.75cm}
	\item \Cle{Epaisseur} := épaisseur des traits (bordures et arcs) ; \hfill~défaut : \Cle{0.75pt}
	\item \Cle{Police} := police globale ; \hfill~défaut : \Cle{\textbackslash footnotesize\textbackslash sffamily}
	\item \Cle{CouleurDates} := couleur des dates, sous la forme \Cle{Couleur} ou \Cle{Couleur\_t/Couleur\_T} ;
	
	\hfill~défaut : \Cle{teal/red}
	\item \Cle{CouleurBords} := couleur des bordures ; \hfill~défaut : \Cle{black}
	\item \Cle{NoirBlanc} := booléen pour tout passer en Noir \&{} Blanc ; \hfill~défaut : \Cle{false}
	\item \Cle{Grille} := pour afficher une grille d'aide (\Cle{\{xmax,ymax\}}), entre (0;\,0) et (xmax;\,ymax).
	
	\hfill~défaut : \Cle{vide}
\end{itemize}

Le deuxième argument, optionnel et entre \texttt{<...>} propose des options, en langage \packagetex!tikz! à passer à l'environnement.
\end{tipblock}

\begin{DemoCode}[]
\begin{GrapheMPM}[Grille={14,5}]
	%commandes
\end{GrapheMPM}
\end{DemoCode}

\pagebreak

\subsection{Arguments et clés pour les tâches}

\begin{DemoCode}[listing only]
\begin{GrapheMPM}[clés]<options tikz>
	\MPMPlaceNotice(*)(coordonnées)
	\MPMPlaceTache(coordonnées)(Tâche)(Dates)
	\MPMPlaceTaches{ (coordA)(TâcheA)(DatesA) / (coordB)(TâcheB)(DatesB) / ... }
\end{GrapheMPM}
\end{DemoCode}

\begin{tipblock}
La commande \motcletex!\MPMPlaceNotice! permet de placer une \textit{notice} :

\begin{itemize}
	\item la version \textit{étoilée} affiche la notice complète, avec les dates et les marges (MT et ML) ;
	\item les coordonnées sont à donner sous la forme \verb!x,y!.
\end{itemize}
\vspace*{-\baselineskip}\leavevmode
\end{tipblock}

\begin{tipblock}
La commande \motcletex!\MPMPlaceTache! permet de placer une tâche :

\begin{itemize}
	\item argument n°1 := coordonnées sont à donner sous la forme \verb!x,y!.
	\item argument n°2 := nom de la tâche, qui sera également le nom du nœud ;
	\item argument n°3 := dates (et marges éventuelles) sous la forme :
	\begin{itemize}
		\item \verb!t,T! pour une tâche présentée de manière \textit{simple} ;
		\item \verb!t,T,MT,ML! pour une tâche présentée de manière \textit{complète} ;
	\end{itemize}
\end{itemize}
\vspace*{-\baselineskip}\leavevmode
\end{tipblock}

\begin{tipblock}
La commande \motcletex!\MPMPlaceTaches! permet de placer plusieurs tâches en utilisant la syntaxe de la commande précédente, les éléments de la liste étant séparés par le caractère \verb!/!.
\end{tipblock}

\begin{DemoCode}[]
\begin{GrapheMPM}[CouleurDates=green/orange,CouleurBords=brown,Grille={18,8}]%
	<scale=0.75,transform shape>
	%NOTICE
	\MPMPlaceNotice(1,6.5)
	%TACHES INDIVIDUELLES
	\MPMPlaceTache(1,4)(Début)(0,0)
	\MPMPlaceTache(3.25,4)(COM)(0,0)
	%TACHES MULTIPLES
	\MPMPlaceTaches{ (5.5,4)(LOG)(1,2) / (5.5,2)(ECR)(1,1) / (5.5,7)(MAT)(1,2{,}5) / (7.75,7)(CABL)(2,4) / (7.75,5.5)(ASS)(2,3{,}5) / (10,4)(INST)(4,5) / (12.25,4)(POST)(7,7) / (14.5,4)(CONF)(8,8) / (16.75,4)(Fin)(9,9) }
\end{GrapheMPM}
\end{DemoCode}

\pagebreak

\subsection{Arguments et clés pour les tâches}

\begin{DemoCode}[listing only]
\begin{GrapheMPM}[clés]<options tikz>
	%DÉCLARATION DES TÂCHES
	\MPMPlaceDuree[clés]{TâcheA>TâcheB,durée}<options tikz>
\end{GrapheMPM}
\end{DemoCode}

\begin{tipblock}
La commande \motcletex!\MPMPlaceDuree! permet de placer un arc avec la durée de la tâche.

\smallskip

La commande propose les \Cle{clés} suivantes :

\begin{itemize}
	\item \Cle{Coude} := booléen pour affiche l'arc sous forme d'un coude ; \hfill~défaut : \Cle{false}
	\item \Cle{SensCoude} := permet de préciser le type de coude, parmi \Cle{HV / VH / VHV} ;
	
	\hfill~défaut : \Cle{HV}
	\item \Cle{HauteurCoude} := dans le cas \Cle{SensCoude=VHV}, permet de préciser le 1\ier{} décalage V ;
	
	\hfill~défaut : \Cle{10pt}
	\item \Cle{DecalHorizDeb} := décalage horizontal du début de l'arc pour la tâche de départ ;
	\item \Cle{DecalVertDeb} := décalage vertical du début de l'arc pour la tâche de départ ;
	\item \Cle{DecalHorizDeb} := décalage horizontal de la fin de l'arc pour la tâche d'arrivée ;
	\item \Cle{DecalVertFin} := décalage vertical de la fin de l'arc pour la tâche d'arrivée.
	
	\hfill~défaut : \Cle{0pt}
\end{itemize}

Le second argument, obligatoire et entre \texttt{\{...\}} permet de spécifier les paramètres de l'arc, sous la forme \verb!TâcheDépart>TâcheArrivée,durée!.

\smallskip

Le troisième argument, optionnel et entre \texttt{<...>} et valant \motcletex!midway! par défaut, permet de spécifier une position différente (en langage \packagetex!tikz!) de la durée (comme par exemple \motcletex!near start!, \motcletex!near end! ou \motcletex!pos=...!).
\end{tipblock}

\begin{noteblock}
Les nœuds créés précédemment permettent donc de spécifier les arguments de la commande, et \textit{tout point d'ancrage} des nœuds peuvent être utilisés pour la commande.

\smallskip

On rappelle que les principaux points d'ancrage d'un nœud \verb!(NOEUD)! \TikZ{} sont :

\begin{itemize}[leftmargin=*]
	\item {\small \verb!(NOEUD.north)!}, {\small \verb!(NOEUD.east)!}, {\small \verb!(NOEUD.south)!}, {\small \verb!(NOEUD.west)!} ;
	\item {\small \verb!(NOEUD.north east)!}, {\small \verb!(NOEUD.south east)!}, {\small \verb!(NOEUD.south west)!}, {\small \verb!(NOEUD.north west)!}.
\end{itemize}
\vspace*{-\baselineskip}\leavevmode
\end{noteblock}

\begin{DemoCode}[text only]
\begin{GrapheMPM}<scale=1.75,transform shape>
	\MPMPlaceTache(3.25,4)(COM)(0,0)
	\foreach \Pos/\Label in {north/above,east/right,south/below,west/left,north east/above right,south east/below right,south west/below left,north west/above left}
		{\filldraw[lightgray] (COM.\Pos) circle[radius=1.75pt] node[font=\scriptsize\ttfamily,\Label] {(COM.\Pos)} ;}
\end{GrapheMPM}
\end{DemoCode}

\begin{warningblock}
Par défaut, les arcs pointent vers le \textit{centre} du nœud, donc dans le cas d'arcs \textit{coudés}, on peut utiliser des points d'ancrage pour une position optimale des arcs.
\end{warningblock}

\pagebreak

\begin{DemoCode}[]
\begin{GrapheMPM}[LargeurCases=0.5cm]<scale=0.9,transform shape>
	%TACHES MULTIPLES
	\MPMPlaceTaches{ (1,4)(Début)(0,0) / (3.25,4)(COM)(0,0) / (5.5,4)(LOG)(1,2) / (5.5,2)(ECR)(1,1) / (5.5,7)(MAT)(1,2{,}5) / (7.75,7)(CABL)(2,4) / (7.75,5.5)(ASS)(2,3{,}5) / (10,4)(INST)(4,5) / (12.25,4)(POST)(7,7) / (14.5,4)(CONF)(8,8) / (16.75,4)(Fin)(9,9) }
	\MPMPlaceDuree{COM>MAT,1}
	\MPMPlaceDuree{COM>LOG,1}\MPMPlaceDuree{COM>ECR,1}
	\MPMPlaceDuree{MAT>CABL,1}\MPMPlaceDuree{MAT>ASS,1}
	\MPMPlaceDuree{LOG>INST,3}<pos=0.85>
	\MPMPlaceDuree[Coude]{ECR>POST,6}<near start>
	\MPMPlaceDuree[Coude]{CABL>CONF,4}<near end>
\end{GrapheMPM}
\end{DemoCode}

\begin{tipblock}
Dans le cas où plusieurs arcs ont les mêmes caractéristiques, on peut utiliser la commande de \textit{placement multiple}, \motcletex!\MPMPlaceDurees!, pour laquelle les \Cle{clés} et l'argument optionnel entre \texttt{<...>} seront passés à \textbf{tous} les arcs.

\smallskip

Dans ce cas, les données sont à spécifier sous forme d'une liste, avec le séparateur \texttt{/}.

\smallskip

Cela permet de \textit{condenser} le code, dans le cas où de multiples arcs ont les mêmes caractéristiques.
\end{tipblock}

\begin{DemoCode}[listing only]
\begin{GrapheMPM}[clés]<options tikz>
	%DÉCLARATION DES TÂCHES
	\MPMPlaceDurees%
		[clés globales]%
		{TâcheA>TâcheB,durée / TâcheC>TâcheD,durée / ... }%
		<options tikz globales>
\end{GrapheMPM}
\end{DemoCode}

\pagebreak

\subsection{Exemples}

\begin{DemoCode}[]
\begin{GrapheMPM}[LargeurCases=0.5cm]<scale=0.9,transform shape>
	%NOTICE
	\MPMPlaceNotice(1,6.5)
	%TÂCHES
	\MPMPlaceTaches{ (1,4)(Début)(0,0) / (3.25,4)(COM)(0,0) / (5.5,4)(LOG)(1,2) / (5.5,2)(ECR)(1,1) / (5.5,7)(MAT)(1,2{,}5) / (7.75,7)(CABL)(2,4) / (7.75,5.5)(ASS)(2,3{,}5) / (10,4)(INST)(4,5) / (12.25,4)(POST)(7,7) / (14.5,4)(CONF)(8,8) / (16.75,4)(Fin)(9,9) }
	%DURÉES (ARCS DIRECTS)
	\MPMPlaceDurees{Début>COM,0 / COM>MAT,1 / COM>LOG,1 / COM>ECR,1 / MAT>CABL,1 / MAT>ASS,1 / LOG>INST,3 / ASS>INST,1{,}5 / INST>POST,2 / POST>CONF,1 / CONF>Fin,1}
	%DURÉES (ARCS COUDÉS)
	\MPMPlaceDurees[Coude]{ECR>POST,6 / CABL>CONF,4}<near start>
\end{GrapheMPM}
\end{DemoCode}

\begin{DemoCode}[]
%ILLUSTRATION DES CLÉS [Decal...]
\begin{GrapheMPM}[CouleurFleches=brown,CouleurDurees=purple,Police=\large\ttfamily]
	%SOMMETS (EXTRAIT)
	\MPMPlaceTaches{ (6.75,2)(O)(10,11) / (15.75,4.5)(N)(26,28) / (20.25,6)(P)(29,29) }
	%ARCS (EXTRAIT)
	\MPMPlaceDuree[Coude,DecalHorizFin=4pt]{O>P.south,8}<near start>
	\MPMPlaceDuree[Coude,SensCoude=VHV,DecalHorizFin=-4pt]{N.south>P.south,1}<near start>
\end{GrapheMPM}
\end{DemoCode}

\pagebreak

\section{Tableau de Karnaugh à trois variables}

\subsection{Commande et fonctionnement global}

\begin{cautionblock}
L'environnement dédié à la création du tableau de Karnaugh est \motcletex!TableKarnaugh!.

C'est en fait un environnement \motcletex!tikzpicture! personnalisé.

\smallskip

Les commandes à utiliser dans l'environnement sont :

\begin{itemize}
	\item \motcletex!\KarnaughCasesResult! ;
	\item \motcletex!\KarnaughBlocRegroup! ;.
\end{itemize}
\vspace*{-\baselineskip}\leavevmode
\end{cautionblock}

\begin{DemoCode}[listing only]
\begin{TableKarnaugh}[clés]<options tikz>
	\KarnaughCasesResult(*){contenu binaire des cases}
	\KarnaughBlocRegroup[clés]{coinA}{coinB}
\end{TableKarnaugh}
\end{DemoCode}
%
%\begin{noteblock}
%Les exemples suivants montrent déjà quelques possibilités de l'environnement, les commandes et clés seront bien évidemment détaillés dans les sous-sections suivantes.
%\end{noteblock}

\begin{DemoCode}[]
\begin{TableKarnaugh}[Aide]
\end{TableKarnaugh}
\hspace{0.5cm}
\begin{TableKarnaugh}[Variables=u/v/w]
	\KarnaughCasesResult{0,1,1,0,1,1,1,1}
	\KarnaughBlocRegroup[Type=Centre,Couleur=blue!75,Decalage=-1.5pt]{10}{32}
	\KarnaughBlocRegroup[Type=Gauche,Couleur=red!75,Decalage=-1.5pt]{00}{11}
	\KarnaughBlocRegroup[Type=Droite,Couleur=red!75,Decalage=-1.5pt]{40}{31}
\end{TableKarnaugh}
\hspace{0.5cm}
\begin{TableKarnaugh}[Variables=u/v/w,Swap]
\end{TableKarnaugh}

\begin{center}
	\begin{TableKarnaugh}[Legende=false,Unite=1.5cm,Epaisseur=1.5pt,Couleur=brown]
	\end{TableKarnaugh}
\end{center}
\end{DemoCode}

\begin{tipblock}
Le tableau créé également des nœuds, qui seront utilisés pour effectuer des \textit{regroupements} de cases, afin de simplifier une expression booléenne.
\end{tipblock}

\pagebreak

\subsection{Arguments et clés pour l'environnement}

\begin{DemoCode}[listing only]
\begin{TableKarnaugh}[clés]<options tikz>
	%commandes
\end{TableKarnaugh}
\end{DemoCode}

\begin{tipblock}
En ce qui concerne la création de l'environnement, les \Cle{clés} sont :

\begin{itemize}
	\item \Cle{Couleur} := couleur du tableau ; \hfill~défaut : \Cle{black}
	\item \Cle{Unite} := unité de base de la figure ; \hfill~défaut : \Cle{1cm}
	\item \Cle{Variables} := nom des variables, sous la forme \Cle{Gauche/Haut/Bas} ; \hfill~défaut : \Cle{a/b/c}
	\item \Cle{Swap} := booléen pour échanger les variables du \textit{bas} ; \hfill~défaut : \Cle{false}
	\item \Cle{Aide} := booléen pour afficher une aide sur les noms des nœuds ; \hfill~défaut : \Cle{false}
	\item \Cle{Epaisseur} := épaisseur des tracés ; \hfill~défaut : \Cle{0.75pt}
	\item \Cle{CouleurCases} := couleur de remplissage des cases ; \hfill~défaut : \Cle{lightgray}
	\item \Cle{CouleurLegende} := couleur de la légende, via \Cle{Couleur} ou \Cle{CouleurA/CouleurB/CouleurC} ;
	
	\hfill~défaut : \Cle{black}
	\item \Cle{PosVarLaterale} := position de la variable \textit{latérale}.\hfill~défaut : \Cle{Gauche}
\end{itemize}

Le deuxième argument, optionnel et entre \texttt{<...>} propose des options, en langage \packagetex!tikz! à passer à l'environnement.
\end{tipblock}

\subsection{Arguments et clés pour la commande de remplissage}

\begin{DemoCode}[listing only]
\begin{TableKarnaugh}[clés]<options tikz>
	\KarnaughCasesResult(*){contenu binaire des cases}
\end{TableKarnaugh}
\end{DemoCode}

\begin{tipblock}
En ce qui concerne le remplissage des cases :

\begin{itemize}
	\item la version \textit{étoilée} permet de \textit{griser} les cases au lieu de les remplir de \texttt{0/1} ;
	\item l'argument obligatoire, et entre \texttt{\{....\}} est la liste des cases, de gauche à droite en partant de la ligne du haut ;
	\item la couleur de cases est gérée par la clé idoine de l'environnement.
\end{itemize}
\vspace*{-\baselineskip}\leavevmode
\end{tipblock}

\subsection{Arguments et clés pour la commande de regroupement des blocs}

\begin{DemoCode}[listing only]
\begin{TableKarnaugh}[clés]<options tikz>
	%remplissage des cases
	\KarnaughBlocRegroup[clés]{coinA}{coinB}
\end{TableKarnaugh}
\end{DemoCode}

\begin{tipblock}
En ce qui concerne le regroupement des cases par blocs, les \Cle{clés} disponibles sont :

\begin{itemize}
	\item \Cle{Couleur} := couleur du \textit{trait} ; \hfill~défaut : \Cle{red}
	\item \Cle{type} := type de regroupement parmi \Cle{Centre/Gauche/Droite} ; \hfill~défaut : \Cle{Centre}
	\item \Cle{Decalage} := décalage du trait par rapports aux cases.\hfill~défaut : \Cle{2pt}
\end{itemize}

Les deux arguments obligatoires, et entre \texttt{\{...\}}, correspondent aux \textit{coins diagonaux} :

\begin{itemize}
	\item sans contrainte pour un rectangle \Cle{Type=Centre} ;
	\item du type \texttt{\{BG\}\{HD\}} pour un rectangle \Cle{Type=Gauche} ;
	\item du type \texttt{\{BD\}\{HG\}} pour un rectangle \Cle{Type=Droite}.
\end{itemize}
\vspace*{-\baselineskip}\leavevmode
\end{tipblock}

\subsection{Exemples}

\begin{DemoCode}[]
\begin{TableKarnaugh}
	\KarnaughCasesResult{0,1,1,0,1,1,1,1}
	\KarnaughBlocRegroup[Type=Centre,Couleur=orange,Decalage=-1.5pt]{10}{32}
	\KarnaughBlocRegroup[Type=Gauche,Couleur=teal,Decalage=-1.5pt]{00}{11}
	\KarnaughBlocRegroup[Type=Droite,Couleur=teal,Decalage=-1.5pt]{40}{31}
\end{TableKarnaugh}
\hspace{5mm}
\begin{TableKarnaugh}[Aide]
	\KarnaughCasesResult{0,1,1,0,1,1,1,1}
	\KarnaughBlocRegroup[Type=Centre,Couleur=orange,Decalage=-1.5pt]{10}{32}
	\KarnaughBlocRegroup[Type=Gauche,Couleur=teal,Decalage=-1.5pt]{00}{11}
	\KarnaughBlocRegroup[Type=Droite,Couleur=teal,Decalage=-1.5pt]{40}{31}
\end{TableKarnaugh}
\hspace{5mm}
\begin{TableKarnaugh}[Swap]
	\KarnaughCasesResult{0,1,1,0,1,1,1,1}
	\KarnaughBlocRegroup[Type=Centre,Couleur=teal,Decalage=-1.5pt]{10}{32}
	\KarnaughBlocRegroup[Type=Gauche,Couleur=orange,Decalage=-1.5pt]{00}{11}
	\KarnaughBlocRegroup[Type=Droite,Couleur=orange,Decalage=-1.5pt]{40}{31}
\end{TableKarnaugh}
\end{DemoCode}

\begin{DemoCode}[]
On obtient le tableau de Karnaugh suivant : 
\begin{TableKarnaugh}
		[Variables=k/l/m,Unite=1.25cm,CouleurCases=cyan!25,Couleur=darkgray, PosVarLaterale=Droite,CouleurLegende=black/blue/red]
		<baseline=(current bounding box.center)>
	\KarnaughCasesResult*{1,1,1,0,1,1,1,0}
	\KarnaughBlocRegroup[Type=Centre,Couleur=brown,Decalage=-3pt]{00}{22}
	\KarnaughBlocRegroup[Type=Centre,Couleur=teal,Decalage=-1.5pt]{10}{32}
\end{TableKarnaugh}
\end{DemoCode}

\pagebreak

\section{Graphes \textit{simples}}

\subsection{Commande et fonctionnement global}

\begin{cautionblock}
L'environnement dédié à la création d'un graphe \textit{simple} est \motcletex!GrapheTikz!.

C'est en fait un environnement \motcletex!tikzpicture! personnalisé.

\smallskip

Les commandes à utiliser dans l'environnement sont :

\begin{itemize}
	\item \motcletex!\GrphPlaceSommets! ;
	\item \motcletex!\GrphTraceAretes! ;.
\end{itemize}
\vspace*{-\baselineskip}\leavevmode
\end{cautionblock}

\begin{DemoCode}[listing only]
\begin{GrapheTikz}[clés]<options tikz>
	\GrphPlaceSommets{liste coordonnées/sommet}
	\GrphTraceAretes(*)[type]<options tikz>{liste arêtes}
\end{GrapheTikz}
\end{DemoCode}

\begin{DemoCode}[]
\begin{GrapheTikz}
	\GrphPlaceSommets{(2,2.5)/A (0,0)/B (5,1)/C}
	\GrphTraceAretes{A/B}
	\GrphTraceAretes[AngleGauche]{C/A}
	\GrphTraceAretes[AngleDroite]{B/C}
	\GrphTraceAretes[Boucle]{A/45 B/135 C/-45}
\end{GrapheTikz}
\end{DemoCode}

\begin{warningblock}
La majorité des paramètres sont personnalisables, mais le \textit{thème} général est globalement \textit{fixé}, dans le sens où ce sont les éléments \textit{cosmétiques} qui pourront être modifiés.

\smallskip

Au contraire du package \packagetex!tkz-graph! qui permet beaucoup plus de choses, les commandes de \packagetex!ProfSio! se veulent beaucoup plus basiques, dans l'optique de travailler avec des graphes en adéquation avec le programme de BTS SIO.
\end{warningblock}

\begin{noteblock}
L'utilisateur pourra également redéfinir les styles utilisées par les commandes de \packagetex!ProfSio! pour refondre le paramétrage global de l'environnement.
\end{noteblock}

\begin{DemoCode}[listing only]
\begin{GrapheTikz}[clés]<options tikz>
	\tikzset{GrphStyleSommet/.style = {...}}
	\tikzset{GrphStyleArc/.style = {...}}
	\tikzset{GrphStylepoids/.style = {...}}
\end{GrapheTikz}
\end{DemoCode}

\begin{tipblock}
La commande de tracé des arêtes nécessite de travailler avec des nœuds existants, donc tout nœud précédemment défini, que ce soit avec la commande de \packagetex!ProfSio! ou tout autre commande pourra être utilisé !
\end{tipblock}

\subsection{Arguments et clés pour l'environnement}

\begin{DemoCode}[listing only]
\begin{GrapheTikz}[clés]<options tikz>
	%commandes
\end{GrapheTikz}
\end{DemoCode}

\begin{tipblock}
En ce qui concerne la création de l'environnement, les \Cle{clés} sont :

\begin{itemize}
	\item \Cle{Police} := police des sommets ; \hfill{}défaut : \Cle{\textbackslash bfseries\textbackslash Large\textbackslash sffamily},%
	\item \Cle{Poids} := police des éventuels poids ; \hfill{}défaut : \Cle{\textbackslash sffamily},%
	\item \Cle{CouleurSommets} := couleur(s) sous la forme \Cle{Couleur} ou \Cle{CouleurBord/CouleurTexte} des sommets ;
	
	\hfill{}défaut : \Cle{black}
	\item \Cle{CouleurFleches} := couleur des arêtes (et des poids) ; \hfill{}défaut : \Cle{black},%
	\item \Cle{TypeSommets} := type de forme des sommets ; \hfill{}défaut : \Cle{circle}
	\item \Cle{Epaisseur} := épaisseur(s) sous la forme \Cle{Epaisseur} ou \Cle{EpaisseurSommet/EpaisseurArête} des traits ;
	
	\hfill{}défaut : \Cle{thick}
	\item \Cle{Unite} := unité globale de la figure ; \hfill{}défaut : \Cle{1cm}
	\item \Cle{CouleurFT} := couleur des arêtes de la fermeture transitive (accessible ensuite via \Cle{FT}) ; \hfill{}défaut : \Cle{black}
	\item \Cle{Grille} := pour afficher une grille d'aide (\Cle{\{xmax,ymax\}}), entre (0;\,0) et (xmax;\,ymax) ;
	
	\hfill~défaut : \Cle{vide}
	\item \Cle{DimensionSommets} := dimension(s) minimale(s) des formes des sommets, sous la forme \Cle{Globale} ou \Cle{Largeur/Hauteur} ;
	
	\hfill{}défaut : \Cle{1cm}
	\item \Cle{PositionFleches} := position, parmi \Cle{Milieu/Fin} pour les flèches ; \hfill{}défaut : \Cle{Fin}
	\item \Cle{EchelleFleches} := échelle de la flèche ; \hfill{}défaut : \Cle{1}
	\item \Cle{TypeFleche} := type (en \TikZ) des flèches.\hfill{}défaut : \Cle{Latex}
\end{itemize}

Le deuxième argument, optionnel et entre \texttt{<...>} propose des options, en langage \packagetex!tikz! à passer à l'environnement.
\end{tipblock}

\pagebreak

\subsection{Arguments et clés pour la commande de création des sommets}

\begin{DemoCode}[listing only]
\begin{GrapheTikz}[clés]<options tikz>
	\GrphPlaceSommets{liste coordonnées/sommet}
\end{GrapheTikz}
\end{DemoCode}

\begin{tipblock}
En ce qui concerne la création des sommets, la liste est à donner sous la forme \verb!(xa,ya)/A (xb,yb)/B (xc,yc)/C ...!

\smallskip

Dans le cas de sommets avec des espaces, il faut les \textit{protéger} par des \texttt{\{...\}}.
\end{tipblock}

\begin{DemoCode}[]
\begin{GrapheTikz}[CouleurSommets={brown/purple},TypeSommets=ellipse,Police={}]
	\GrphPlaceSommets{(2,2.5)/Hôpital (0,0)/Mairie (5,1)/Banque}
\end{GrapheTikz}
\end{DemoCode}

\begin{DemoCode}[]
\begin{GrapheTikz}[Epaisseur={very thick},Grille={5,4},DimensionSommets=1.5cm]
	\GrphPlaceSommets{(0,0)/K (4,0)/L (60:4)/M}
\end{GrapheTikz}
\end{DemoCode}

\begin{DemoCode}[]
\begin{GrapheTikz}[TypeSommets=diamond,DimensionSommets=2cm/1.5cm]
	\GrphPlaceSommets{(0,0)/K (4,0)/L (60:4)/M}
\end{GrapheTikz}
\end{DemoCode}

\pagebreak

\subsection{Arguments et clés pour la commande de tracé des arêtes}

\begin{DemoCode}[listing only]
\begin{GrapheTikz}[clés]<options tikz>
	%commandes de placement des sommets
	\GrphTraceAretes(*)[type]<options tikz>{liste arêtes}
\end{GrapheTikz}
\end{DemoCode}

\begin{tipblock}
En ce qui concerne le tracés des arêtes, la commande permet de tracer des arêtes ayant le même style.

\smallskip

La version \textit{étoilée} permet de pondérer l'arête (le poids est, par défaut, situé sur le milieu de l'arête).

\smallskip

Les \Cle{type} d'arête, disponible entre \texttt{[...]} et valant \Cle{Droit} par défaut, de la commande peut valoir :

\begin{itemize}
	\item \Cle{Droit} := permet de tracer des arêtes orientées \textit{droites} ;
	\item \Cle{AngleGauche} ou \Cle{AngleGauche=...} := permet de tracer des arêtes orientées \textit{courbées vers la gauche}, avec par défaut un angle de 10° ;
	\item \Cle{AngleDroite} ou \Cle{AngleDroite=...} := permet de tracer des arêtes orientées \textit{courbées vers la droite}, avec par défaut un angle de 10° ;
	\item \Cle{Boucle} ou \Cle{Boucle=...} := permet de tracer une boucle avec un coefficient \motcletex!looseness! de 6 par défaut.
\end{itemize}

Dans le cas d'arêtes \textit{classiques}, la liste est à donner sous la forme \verb|Deb/Fin Deb/Fin Deb/Fin ...| ou \verb|Deb/Fin/Poids Deb/Fin/Poids Deb/Fin/Poids ...|

\smallskip

Dans le cas de boucles, la lise est à donner sous la forme \verb|Som/anglesortie Som/anglesortie ...| ou \verb|Som/anglesortie/Poids Som/anglesortie/Poids ...| en sachant que (par défaut) l'angle d'entrée est fixé 90° après dans le sens trigonométrique.

\smallskip

Pour marquer une fermeture transitive, on peut utiliser le style \verb|FT| dans les \textit{options tikz} de la commande.
\end{tipblock}

\begin{DemoCode}[]
\begin{GrapheTikz}
	\GrphPlaceSommets{(0,0)/A (3,1)/B}
	\GrphTraceAretes{A/B}
\end{GrapheTikz}
\hspace{5mm}
\begin{GrapheTikz}
	\GrphPlaceSommets{(0,0)/A (3,1)/B}
	\GrphTraceAretes[AngleGauche]{A/B}
\end{GrapheTikz}
\hspace{5mm}
\begin{GrapheTikz}
	\GrphPlaceSommets{(0,0)/A (3,1)/B}
	\GrphTraceAretes*[AngleGauche]{A/B/10}
\end{GrapheTikz}
\end{DemoCode}

\begin{DemoCode}[]
\begin{GrapheTikz}
	\GrphPlaceSommets{(0,0)/A (3,1)/B}
	\GrphTraceAretes[AngleDroite=45]{A/B}
\end{GrapheTikz}
\hspace{5mm}
\begin{GrapheTikz}
	\GrphPlaceSommets{(0,0)/A (3,1)/B}
	\GrphTraceAretes[AngleGauche]{A/B B/A}
\end{GrapheTikz}
\end{DemoCode}

\begin{DemoCode}[]
\begin{GrapheTikz}
	\GrphPlaceSommets{(0,0)/A (4,1)/B (2,3)/C}
	\GrphTraceAretes[Boucle]{A/135 B/15}
	\GrphTraceAretes*[Boucle=10]{C/60/{0{,}3}}
\end{GrapheTikz}
\end{DemoCode}

\subsection{Exemples}

\begin{DemoCode}[]
\begin{GrapheTikz}
	[DimensionSommets=14pt,Police=\bfseries\sffamily,CouleurSommets={blue/orange}]
	%SOMMETS
	\GrphPlaceSommets{(1,4)/A (3,4)/B (4,3)/C (3,2)/D (2,1)/E (0,1)/F}
	%ARÊTES
	\GrphTraceAretes{A/B B/D B/E C/A C/D D/A D/E E/F F/B}
\end{GrapheTikz}
\end{DemoCode}

\begin{DemoCode}[]
\begin{GrapheTikz}
	%SOMMETS
	\GrphPlaceSommets{(1,6)/A (7,6)/B (7,3)/C (4,1)/D (1,3)/E}
	%ARÊTES
	\GrphTraceAretes[AngleGauche]{A/B A/D A/E B/E C/E E/C D/C C/D E/D D/E}
	%FT
	\GrphTraceAretes[AngleGauche]<FT>{A/C B/C B/D}
	\GrphTraceAretes[Boucle]<FT>{C/-45 E/135 D/-135}
\end{GrapheTikz}
\end{DemoCode}

\pagebreak

\section{Diagramme sagittal d'une application}

\subsection{Commande et fonctionnement global}

\begin{cautionblock}
La commande dédiée à la création d'un diagramme sagittal pour une application est \motcletex!\DiagrammeSagittal!.

Le diagramme créé est réalisé avec un environnement \motcletex!tikzpicture!.
\end{cautionblock}

\begin{DemoCode}[listing only]
%commande autonome
\DiagrammeSagittal[clés]<options tikz>{liaisons}

%commande à insérer dans un environnement tikzpicture
\begin{tikzpicture}
	\DiagrammeSagittal*[clés]{liaisons}
\end{tikzpicture}
\end{DemoCode}

\begin{DemoCode}[]
\DiagrammeSagittal[E={a,b,c},F={A,C,H,P}]{a/A,a/P,b/H,b/P,c/C}
\end{DemoCode}

\begin{warningblock}
La majorité des paramètres sont personnalisables, mais le \textit{thème} général est globalement \textit{fixé}, dans le sens où ce sont les éléments \textit{cosmétiques} qui pourront être modifiés.

\smallskip

La commande de création de \packagetex!ProfSio! est volontairement pour des applications basiques, dans l'optique de travailler avec exemples en adéquation avec le programme de BTS SIO.
\end{warningblock}

\subsection{Arguments et clés}

\begin{DemoCode}[listing only]
\DiagrammeSagittal[clés]<options tikz>{liaisons}

\begin{tikzpicture}
	\DiagrammeSagittal*[clés]{liaisons}
\end{tikzpicture}
\end{DemoCode}

\begin{noteblock}
Le code se charge, grâce aux \Cle{clés}, de positionner et d'aligner les éléments des ensembles et les flèches.

De ce fait, les \textit{écarts} entre les éléments d'un ensemble sont fixées globalement, tout comme le style général des flèches.
\end{noteblock}

\begin{tipblock}
La version \textit{étoilé} permet de ne pas créer l'environnement \motcletex!tikzpicture!, pour d'éventuels rajouts ultérieurs :

\begin{itemize}
	\item les éléments de l'ensemble de départ sont des nœuds nommés \verb!(E...)! ;
	\item les éléments de l'ensemble d'arrivée sont des nœuds nommés \verb!(F...)!.
\end{itemize}
\vspace*{-\baselineskip}\leavevmode
\end{tipblock}

%\begin{DemoCode}[]
%\DiagrammeSagittal[E={A,B,C,D,E,F},F={1,2,3,4}]{}
%\end{DemoCode}

\begin{tipblock}
Les \Cle{clés} disponibles sont :

\begin{itemize}
	\item \Cle{DistElem} := distance verticale entre les éléments ; \hfill{}défaut : \Cle{0.75}
	\item \Cle{DistEns} := distance entre les \og patates \fg{} ; \hfill{}défaut : \Cle{4}
	\item \Cle{LargEns} := largeur des \og patates \fg{} ; \hfill{}défaut : \Cle{1.5}
	\item \Cle{NomAppli} := nom de l'application ; \hfill{}défaut : \Cle{\$f\$}
	\item \Cle{CouleurE} := couleur de l'ensemble de départ ; \hfill{}défaut : \Cle{blue}
	\item \Cle{CouleurAppli} := couleur de l'application ; \hfill{}défaut : \Cle{violet}
	\item \Cle{CouleurF} := couleur de l'ensemble d'arrivée ; \hfill{}défaut : \Cle{red}
	\item \Cle{CouleurFleches} := couleur des flèches ; \hfill{}défaut : \Cle{teal}
	\item \Cle{TypeFleche} := type de la flèche  ; \hfill{}défaut : \Cle{Latex}
	\item \Cle{Offset} := décalage entre les flèches et les éléments (points) ; \hfill{}défaut : \Cle{2pt}
	\item \Cle{Epaisseur} := épaisseur des tracés ; \hfill{}défaut : \Cle{0.8pt}
	\item \Cle{Police} := police pour les éléments ; \hfill{}défaut : \Cle{vide}
	\item \Cle{NoirBlanc} := booléen pour forcer l'affichage en N\&{}B ; \hfill{}défaut : \Cle{false}
	\item \Cle{Labels} := booléen pour afficher les noms des ensembles ; \hfill{}défaut : \Cle{true}
	\item \Cle{Ensembles} := nom des ensembles  ; \hfill{}défaut : \Cle{\$\textbackslash mathcal\{E\}\$/\$\textbackslash  mathcal\{F\}\$}
	\item \Cle{PosLabels} := position des labels, parmi \Cle{haut/bas}. \hfill{}défaut : \Cle{bas}
\end{itemize}

Le deuxième argument, optionnel et entre \texttt{<...>} propose des options, en langage \packagetex!tikz! à passer à l'environnement.

\smallskip

Le troisième argument, obligatoire et entre \texttt{\{...\}}, permet de préciser les \textit{liaisons} sous la forme \verb!x1/f(x1),x2/f(x2),...!.
\end{tipblock}

\subsection{Exemples}

\begin{DemoCode}[]
\DiagrammeSagittal[DistElem=1,DistEns=5,LargEns=1.75,Police={\Large\ttfamily}, Epaisseur=1pt,NomAppli={$h$},E={a,b,c},F={A,C,H,P}, PoliceLabels=\Large]{a/A,a/P,b/H,b/P,c/C}
\end{DemoCode}

\begin{DemoCode}[]
\DiagrammeSagittal[%
	E={1,2,3,4,5,6,7},F={a,b,c,d,e},Labels=false,%
	DistElem=1,DistEns=6,LargEns=2,Offset=4pt,%
	CouleurE=teal,CouleurF=orange,CouleurAppli=brown,CouleurFleches=brown
	]{1/a,2/b,3/b,4/c,5/d,6/e,7/e}
\end{DemoCode}

\begin{DemoCode}[]
\begin{tikzpicture}
	\DiagrammeSagittal*[%
	E={1,2,3,4,5,6,7},F={a,b,c,d,e},DistEns=6,%
	CouleurE=teal,CouleurF=orange,CouleurAppli=brown,CouleurFleches=brown
	]{1/a,2/b,3/b,4/c,5/d,6/d,7/d}
	\draw[lime!50!black,<-,thick,dashed,>=Latex] ($(Fb)+(12pt,0)$) to[bend right=10]++ (2,1) node[right] {\parbox{4cm}{b admet 2 antécédents, donc $f$ ne peut pas être injective.}} ;
	\draw[blue!50!black,<-,thick,dashed,>=Latex] ($(Fe)+(12pt,0)$) to[bend left=10]++ (2,-1) node[right] {\parbox{4cm}{e n'admet pas d'antécédent, donc $f$ ne peut pas être surjective.}} ;
\end{tikzpicture}
\end{DemoCode}

\pagebreak

\section{Diagramme sagittal d'une composée d'applications}

\subsection{Commande et fonctionnement global}

\begin{cautionblock}
La commande dédiée à la création d'un diagramme sagittal pour une application est \motcletex!\DiagrammeSagittalCompo!.

Le diagramme créé est réalisé avec un environnement \motcletex!tikzpicture!.
\end{cautionblock}

\begin{DemoCode}[listing only]
%commande autonome
\DiagrammeSagittalCompo[clés]<options tikz>{liaisons1}{liaisons2}

%commande à insérer dans un environnement tikzpicture
\begin{tikzpicture}
	\DiagrammeSagittalCompo*[clés]{liaisons1}{liaisons2}
\end{tikzpicture}
\end{DemoCode}

\begin{DemoCode}[]
\DiagrammeSagittalCompo%
	[E={a,b,c,d},F={A,B,C,E,F,G,H},G={1,2,3,4,5}]%
	{a/B,d/H}%
	{B/1,B/2,H/5}
\end{DemoCode}

\begin{warningblock}
La majorité des paramètres sont personnalisables, mais le \textit{thème} général est globalement \textit{fixé}, dans le sens où ce sont les éléments \textit{cosmétiques} qui pourront être modifiés.

\smallskip

La commande de création de \packagetex!ProfSio! est volontairement pour des applications basiques, dans l'optique de travailler avec exemples en adéquation avec le programme de BTS SIO.
\end{warningblock}

\subsection{Arguments et clés}

\begin{DemoCode}[listing only]
\DiagrammeSagittalCompo[clés]<options tikz>{liaisons1}{liaisons2}

\begin{tikzpicture}
	\DiagrammeSagittalCompo*[clés]{liaisons1}{liaisons2}
\end{tikzpicture}
\end{DemoCode}

\begin{noteblock}
Le code se charge, grâce aux \Cle{clés}, de positionner et d'aligner les éléments des ensembles et les flèches.

De ce fait, les \textit{écarts} entre les éléments d'un ensemble sont fixées globalement, tout comme le style général des flèches.
\end{noteblock}

\begin{tipblock}
La version \textit{étoilé} permet de ne pas créer l'environnement \motcletex!tikzpicture!, pour d'éventuels rajouts ultérieurs :

\begin{itemize}
	\item les éléments de l'ensemble de départ sont des nœuds nommés \verb!(E...)! ;
	\item les éléments de l'ensemble du milieu sont des nœuds nommés \verb!(F...)! ;
	\item les éléments de l'ensemble d'arrivée sont des nœuds nommés \verb!(G...)!.
\end{itemize}
\vspace*{-\baselineskip}\leavevmode
\end{tipblock}

\begin{tipblock}
Les \Cle{clés} disponibles sont :

\begin{itemize}
	\item \Cle{DistElem} := distance verticale entre les éléments ; \hfill{}défaut : \Cle{0.75}
	\item \Cle{DistEns} := distance entre les \og patates \fg{} ; \hfill{}défaut : \Cle{4}
	\item \Cle{LargEns} := largeur des \og patates \fg{} ; \hfill{}défaut : \Cle{1.5}
	\item \Cle{NomApplis} := nom des applications ; \hfill{}défaut : \Cle{\$f\$/\$g\$}
	\item \Cle{CouleurE} := couleur de l'ensemble de départ ; \hfill{}défaut : \Cle{blue}
	\item \Cle{CouleurApplis} := couleurs des applications, \Cle{Couleur} ou \Cle{Couleur\_f/Couleur\_g} ;
	
	\hfill{}défaut : \Cle{violet}
	\item \Cle{CouleurF} := couleur de l'ensemble du milieu ; \hfill{}défaut : \Cle{red}
	\item \Cle{CouleurG} := couleur de l'ensemble d'arrivée ; \hfill{}défaut : \Cle{gray}
	\item \Cle{CouleursFleches} := couleurs des flèches, \Cle{Couleur} ou \Cle{Couleur\_f/Couleur\_g} ; 
	
	\hfill{}défaut : \Cle{teal}
	\item \Cle{TypeFleche} := type de la flèche  ; \hfill{}défaut : \Cle{Latex}
	\item \Cle{Offset} := décalage entre les flèches et les éléments (points) ; \hfill{}défaut : \Cle{2pt}
	\item \Cle{Epaisseur} := épaisseur des tracés ; \hfill{}défaut : \Cle{0.8pt}
	\item \Cle{Police} := police pour les éléments ; \hfill{}défaut : \Cle{vide}
	\item \Cle{NoirBlanc} := booléen pour forcer l'affichage en N\&{}B ; \hfill{}défaut : \Cle{false}
	\item \Cle{Labels} := booléen pour afficher les noms des ensembles ; \hfill{}défaut : \Cle{true}
	\item \Cle{Ensembles} := nom des ensembles  ;
	
	\hfill{}défaut : \Cle{\$\textbackslash mathcal\{E\}\$/\$\textbackslash  mathcal\{F\}\$/\$\textbackslash  mathcal\{G\}\$}
	\item \Cle{PosLabels} := position des labels, parmi \Cle{haut/bas}. \hfill{}défaut : \Cle{bas}
\end{itemize}

Le deuxième argument, optionnel et entre \texttt{<...>} propose des options, en langage \packagetex!tikz! à passer à l'environnement.

\smallskip

Les arguments \texttt{3} et \texttt{4}, obligatoires et entre \texttt{\{...\}}, permettent de préciser les \textit{liaisons} sous la forme \verb!x1/f(x1),x2/f(x2),...! et \verb!y1/g(y1),y2/g(y2),...! .
\end{tipblock}

\subsection{Exemples}

\begin{DemoCode}[]
\DiagrammeSagittalCompo%
	[DistElem=1,DistEns=5,LargEns=1.75,Police={\Large\ttfamily},%
	Epaisseur=1pt,NomApplis={$h$/$\varphi$},CouleursFleches={teal/lime},%
	E={a,b,c,d,e,f},F={A,C,H,P},G={1,2,3,4,5},PoliceLabels=\Large]%
	{a/A,a/P,b/H,e/P,c/C}%
	{A/5,C/3,H/3,P/4}%
\end{DemoCode}

\begin{DemoCode}[]
\DiagrammeSagittalCompo%
	[E={a,b,c,d,e},F={1,2,3,4,5,6,7},G={a,b,c,d,e},%
	Ensembles={E/F/E}]%
	{a/1,b/2,c/4,d/5,e/6}%
	{1/a,2/b,3/b,4/c,5/d,6/e,7/e}%
\end{DemoCode}

\begin{DemoCode}[]
\begin{tikzpicture}
	\DiagrammeSagittalCompo*%
		[E={a,b,c,d,e},F={1,2,3,4,5,6,7},G={a,b,c,d,e},%
		Ensembles={E/F/E}]%
		{a/1,b/2,c/4,d/5,e/6}%
		{1/a,2/b,3/b,4/c,5/d,6/e,7/e}%
	\draw[orange!50!black,<-,thick,dashed,>=Latex] ($(Gc)+(12pt,0)$) to[bend left=10]++ (2,-1) node[right] {\parbox{4cm}{On a donc $g\,{\small\circ\,}f(c)=c$}} ;
\end{tikzpicture}
\end{DemoCode}

\pagebreak

\section{Table de vérité}

\subsection{Commande et fonctionnement global}

\begin{cautionblock}
La commande dédiée à la création d'une table de vérité (à deux variables minimum) est \motcletex!\TableVerite!.

La commande est accessible \textbf{uniquement} en cas d'une compilation en \hologo{LuaLaTeX} !

Le tableau est créé grâce au package \packagetex!nicematrix!.
\end{cautionblock}

\begin{importantblock}
Une \textbf{double compilation} peut être nécessaire pour le placement correct des filets !

Les fonction \textsf{LUA} utilisées sont issues du \packagetex!luatruthtable!, elles sont légèrement modifiées pour \textit{coller} à une présentation plus classique.
\end{importantblock}

\begin{DemoCode}[listing only]
\TableVerite[clés]<opts nicematrix>{vars}{colonnes_vars}{calculs}{colonnes_calculs}
\end{DemoCode}

\begin{DemoCode}[]
\TableVerite{P}{$P$}%
	{lognot*P,P*logand*P,P*logor*P,P*iff*P,P*imp*P}%
	{$\lnot P$,$P \land P$,$P \lor P$,$P \Leftrightarrow P$,$P \Rightarrow P$}
\end{DemoCode}

\begin{DemoCode}[]
\TableVerite{P,Q}{$P$,$Q$}{lognot*P,P*logand*Q}{$\lnot P$,$P \land Q$}
\end{DemoCode}

\subsection{Arguments et clés pour la commande}

\begin{DemoCode}[listing only]
\TableVerite[clés]<opts nicematrix>{vars}{colonnes_vars}{calculs}{colonnes_calculs}
\end{DemoCode}

\begin{tipblock}
En ce qui concerne la création du tableau, les \Cle{clés} sont :

\begin{itemize}
	\item \Cle{VF} := pour préciser Vrai/Faux ; \hfill~défaut : \Cle{V/F}
	\item \Cle{Swap} := booléen pour commencer par FFF au lieu de VVV ; \hfill~défaut : \Cle{false}
	\item \Cle{LargeursColonnes} := largeur des colonnes, \Cle{auto} ou \Cle{largeurG} ou \Cle{LargeurVar/LargeurCalc} ;
	
	\hfill~défaut : \Cle{auto}
	\item \Cle{CouleurEnonce} := couleur de fond de la première ligne ; \hfill~défaut : \Cle{vide}
	\item \Cle{CodeAvant} := code \texttt{CodeBefore} (et avant l'éventuel coloriage de la première ligne) pour \motcletex!nicetabular! ;
	
	\hfill~défaut : \Cle{vide}
	\item \Cle{CodeApres} := code \texttt{CodeAfter} pour \motcletex!nicetabular!. \hfill~défaut : \Cle{vide}
\end{itemize}

Le deuxième argument, optionnel et entre \texttt{<...>} propose des options, en langage \packagetex!nicematrix! à passer à la commande.
\end{tipblock}

\pagebreak

\begin{tipblock}
Le troisième argument, obligatoire et entre \texttt{\{...\}}, permet de spécifier les calculs à effectuer, en langage \motcletex!luatruthtable!, notamment :

\begin{itemize}
	\item \texttt{lognot*} pour le \textsf{CONTRAIRE} ;
	\item \texttt{*logand*} pour le \textsf{ET} ;
	\item \texttt{*logor*} pour le \textsf{OU} ;
	\item \texttt{*iff*} pour le \textsf{ÉQUIVALENT} ;
	\item \texttt{*imp*} pour le \textsf{IMPLIQUE} ;
	\item le reste est disponible dans la documentation (\url{http://mirrors.ctan.org/macros/luatex/latex/luatruthtable/luatruthtable.pdf}).
\end{itemize}

Le dernier argument, obligatoire et entre \texttt{\{...\}}, permet de spécifier les labels des calculs, en langage \LaTeX{} cette fois-ci.
\end{tipblock}

\subsection{Compléments pour le package luatruthtable}

\begin{tipblock}
Le tableau suivant présente les connecteurs logiques issues du package \packagetex!luatruthtable! :

\begin{center}
	\begin{tblr}{colspec={Q[m,l]Q[m,l]Q[m,l]Q[m,l]},row{1}={fg=red!50!black,font=\sffamily},cell{2-Z}{1-2}={fg=cyan!75!black},column{3}={c}}
		\toprule
		Opérateur & Syntaxe & Expression & Description \\
		\toprule
		\texttt{lognot*} & \texttt{lognot*P} & $\lnot P$ & Négation de P \\
		\midrule
		\texttt{*logand*} & \texttt{P*logand*Q} & $P \land Q$ &  Conjonction (et) de P et Q \\
		\midrule
		\texttt{*logor*} & \texttt{P*logor*Q} & $P \lor Q$ & Disjonction (ou) de P et Q \\
		\midrule
		\texttt{*imp*} & \texttt{P*imp*Q} & $P \Rightarrow Q$ & Implication de P vers Q \\
		\midrule
		\texttt{*iff*} & \texttt{P*iff*Q} & $P \Leftrightarrow Q$ & Équivalence de P et Q \\
		\midrule
		\texttt{*lognand*} & \texttt{P*lognand*Q} & $\lnot(P \land Q)$ & NAND de P et Q \\
		\midrule
		\texttt{*logxor*} & \texttt{P*logxor*Q} & $(P \lor Q) \land \lnot (P \land Q)$ & XOR de P et Q \\
		\midrule
		\texttt{*lognor*} & \texttt{P*lognor*Q} & $\lnot(P \lor Q)$ & NOR de P et Q \\
		\midrule
		\texttt{*logxnor*} & \texttt{P*logxnor*Q} & $(P \land Q) \lor (\lnot P \land \lnot Q)$ & XNOR de P et Q \\
		\bottomrule
	\end{tblr}
\end{center}
\end{tipblock}

\subsection{Exemples}

\begin{DemoCode}[]
\TableVerite{P,Q}{$P$,$Q$}{lognot*P,P*logand*Q}{$\lnot P$,$P \land Q$}
\end{DemoCode}

\begin{DemoCode}[]
\TableVerite[LargeursColonnes=2cm]{P,Q}{$P$,$Q$}{lognot*P,P*logand*Q}{$\lnot P$,$P \land Q$}
\end{DemoCode}

\begin{DemoCode}[]
\TableVerite[LargeursColonnes=1cm/2cm]{P,Q}{$P$,$Q$}{lognot*P,P*logand*Q}{$\lnot P$,$P \land Q$}
\end{DemoCode}

\begin{DemoCode}[]
\TableVerite[CouleurEnonce=lightgray!25]{P,Q}{$P$,$Q$}{lognot*P,P*logand*Q}{$\lnot P$,$P \land Q$}
\end{DemoCode}

\begin{DemoCode}[]
\TableVerite%
	[CodeAvant={\columncolor{red!15}{1}\columncolor{teal!15}{4}}]%
	{P,Q}{$P$,$Q$}{lognot*P,P*logand*Q}{$\lnot P$,$P \land Q$}
\end{DemoCode}

\begin{DemoCode}[]
\TableVerite%
	[CodeApres={\UnderBrace[yshift=4pt]{1-4}{5-4}{et}}]%
	{P,Q}{$P$,$Q$}{lognot*P,P*logand*Q}{$\lnot P$,$P \land Q$}
\hspace{5mm}
\TableVerite%
	[Swap,CodeApres={\UnderBrace[yshift=4pt]{1-4}{5-4}{et}}]%
	{P,Q}{$P$,$Q$}{lognot*P,P*logand*Q}{$\lnot P$,$P \land Q$}
\hspace{5mm}
\TableVerite%
	[VF={Vrai/Faux},CodeApres={\UnderBrace[yshift=4pt]{1-4}{5-4}{et}}]%
	{P,Q}{$P$,$Q$}{lognot*P,P*logand*Q}{$\lnot P$,$P \land Q$}
\vspace*{0.75cm}
\end{DemoCode}

\begin{DemoCode}[]
\TableVerite%
	[CodeAvant={\columncolor{red!15}{5}\columncolor{red!15}{8}}]%
	{P,Q,R}%
	{$P$,$Q$,$R$}%
	{%
		Q*logand*R,P*logor*(Q*logand*R),P*logor*Q,%
		Q*logor*R,(P*logor*Q)*logand*(P*logor*R)
	}%
	{$Q \land R$,$P \lor (Q \land R)$,$P \lor Q$,$Q \lor R$,$(P \lor Q) \land (P \lor R)$}
\end{DemoCode}

\begin{DemoCode}[]
%Loi de De Morgan

\TableVerite%
	[CouleurEnonce=lightgray!15,LargeursColonnes=0.75cm/2cm,%
	CodeAvant={\columncolor{teal!10}{6}\columncolor{teal!10}{7}}]%
	{P,Q}{$P$,$Q$}%
	{lognot*P,lognot*Q,P*logand*Q,
		lognot*(P*logand*Q),(lognot*P)*logor*(lognot*Q)}%
	{$\lnot P$,$\lnot Q$,$P\land Q$,$\lnot(P\land Q)$,$(\lnot P)\lor(\lnot Q)$}

\TableVerite%
	[CouleurEnonce=lightgray!15,LargeursColonnes=0.75cm/2cm,VF={1/0},%
	CodeAvant={\columncolor{orange!10}{6}\columncolor{orange!10}{7}}]%
	{P,Q}{$P$,$Q$}%
	{lognot*P,lognot*Q,P*logand*Q,
		lognot*(P*logand*Q),(lognot*P)*logor*(lognot*Q)}%
	{$\lnot P$,$\lnot Q$,$P\land Q$,$\lnot(P\land Q)$,$(\lnot P)\lor(\lnot Q)$}
\end{DemoCode}

\end{document}